\documentclass[review]{elsarticle}

\usepackage{lineno,hyperref}
\modulolinenumbers[5]

\journal{Heliyon}

%%%%%%%%%%%%%%%%%%%%%%%
%% Elsevier bibliography styles
%%%%%%%%%%%%%%%%%%%%%%%
%% To change the style, put a % in front of the second line of the current style and
%% remove the % from the second line of the style you would like to use.
%%%%%%%%%%%%%%%%%%%%%%%

%% Numbered
%\bibliographystyle{model1-num-names}

%% Numbered without titles
%\bibliographystyle{model1a-num-names}

%% Harvard
%\bibliographystyle{model2-names.bst}\biboptions{authoryear}

%% Vancouver numbered
%\usepackage{numcompress}\bibliographystyle{model3-num-names}

%% Vancouver name/year
%\usepackage{numcompress}\bibliographystyle{model4-names}\biboptions{authoryear}

%% APA style
%\bibliographystyle{model5-names}\biboptions{authoryear}

%% AMA style
%\usepackage{numcompress}\bibliographystyle{model6-num-names}

%% `Elsevier LaTeX' style
\bibliographystyle{elsarticle-num}
%%%%%%%%%%%%%%%%%%%%%%%
%George's Additions
%%%%%%%%%%%%%%%%%%%%%%%
\usepackage{subcaption}
\usepackage{changepage}
\usepackage{amsmath}
\usepackage{mathtools}
\usepackage{amsmath,amssymb}
\newtheorem{theorem}{Theorem}[section]
\newtheorem{proposition}{Proposition}[section]
\newtheorem{corollary}{Corollary}[theorem]
\newtheorem{lemma}[theorem]{Lemma}

%% Misc. Packages
\usepackage[final]{changes}
\definechangesauthor[name={George Chacko}, color=orange]{gc}
\setremarkmarkup{(#2)}
\usepackage{listings}


\begin{document}

\begin{frontmatter}

\title{ERNIE: Accessible Support for Research Assessment} 
%\tnotetext[mytitlenote]{This document is a collaborative effort}

%% Group authors per affiliation:
%% or include affiliations in footnotes:
\author[nl]{Samet Keserci}
\author[nl]{Avon Davey}
\author[ca]{Di Cross}
\author[gi]{Alexander R. Pico}
\author[nl]{Dmitriy Korobskiy}
\author[nl]{George Chacko \corref{cor1}}
\ead{netelabs@nete.com}

\cortext[cor1]{Corresponding author}
%\fntext[fn1]{Current address: Facebook Inc., Menlo Park, CA, USA}


\address[nl]{NETE Labs, NET ESolutions Corporation, McLean, VA, USA}
\address[ca]{Research Data Science \& Evaluation, Clarivate Analytics, USA}
\address[gi]{Gladstone Institutes, San Francisco, CA, USA}

\raggedright

\begin{abstract}

Data mining of public and commercially available data sources coupled with network analysis has been successfully used as a digital methodology to document and evaluate research collaborations and knowledge flow associated with drug development. To support quantitative studies based on this approach, we have developed Enhanced Research Network Information Environment (ERNIE), a scalable cloud-based knowledge platform that integrates free data drawn from public sources as well as licensed data from commercially available ones. To facilitate adoption, reuse and extensibility, ERNIE was built with open source tools. and a modular design enables the facile addition, deletion, or substitution of data sources. Analytical workflows in ERNIE are partially automated to enable expert input at critical stages. To demonstrate the capabilities of ERNIE, we report the results of seven case studies that span drug development, pharmacogenomics, target discovery, behavioral interventions, and opioid addiction. In these studies, we mine and analyze data from policy documents, regulatory approvals, research grants, bibliographic and patent databases, and clinical trials, to document collaborations and identify influential research accomplishments. ERNIE is a template for repositories that can be used to support expert qualitative assessments, while offering burden reduction through automation and access to integrated data.

\end{abstract}

%\begin{keyword}
%\end{keyword}

\end{frontmatter}       

\linenumbers
\raggedright

\section*{Introduction} 

Given the risk of the research evaluation being driven by data and metrics without sufficiently considering balanced judgment, sound principles have been articulated to advise its practice~\cite{LeidenManifesto2015}. However, data are still critical and utilitarian approaches for capture, integration, and archival that offer coverage and burden reduction assist the practice of research evaluation especially when coupled with improved analytical methods. While bibliographic data is usually central to research evaluation, the use of administrative records such federal government statistics and those of research institutions and research funders can be valuable as a supplemental source~\cite{FedStat2017}. 

Using data mining of publicly available administrative, regulatory, and scientific records and network analysis, Williams and colleagues defined relationships between scientific discoveries and major advances in medicine such as new drugs~\cite{Williams2015}. Extending this work, we have previously documented the scientific collaborations that extend across networks underlying the development of five independently developed therapeutics for cancer~\cite{Keserci2017}. Analysis of such networks supports new insights into collaboration, knowledge diffusion, and recursive learning \textit{(supra vide)}. The framework we developed to mine data and integrate it into networks for analysis has the benefit of retaining expert inputis flexible but can be further automated, adapted to a range of subjects beyond drug development, and used to support individual researchers, groups, and research organizations engaged in research assessments. In pursuit of these objectives we have developed Enhanced Research Network Informatics Environment (ERNIE), a knowledge platform that integrates publicly available data as well as data from commercial sources. Since cloud computing has evolved to the point where scalable computing services and storage are accessible to a broad user population, ERNIE was designed to exist as a cloud resource. 

Focusing on the theme of data mining and network analysis, we have used the data within ERNIE to conduct seven case studies. In these studies, we mine and analyze data from policy documents, regulatory approvals, research grants, bibliographic and patent databases, and clinical trials, to document collaborations and identify influential research accomplishments. The workflows used incorporate expert knowledge into defining a set of core documents from which citations can be extracted, linked to other records in ERNIE, and analyzed. The first two case studies serve to validate ERNIE through reproducing the results of prior studies on ivacaftor and ipilimumab, breakthrough drugs used to treat cystic fibrosis and melanoma~\cite{Williams2015}. The remaining case studies concern drugs used to treat opioid addiction, a microarray system for pharmacogenomic profiling, a target discovery system based on enzyme fragment complementation assays, and a behavioral intervention for substance abuse. The datasets that result from these studies are made available for use by other researchers and the workflows used to generate them are also archived to permit reproduction of our results, as well as review and modification of the methods used to generate them. 

\section*{Materials and Methods}

\emph{Infrastructure} ERNIE exists as two Centos 7.4 virtual machines in the Microsoft Azure cloud with 10 terabytes of attached storage. One virtual machine,  standard D8s v3 (8 vcpu, 32 Gb), houses data in a PostgreSQL 9.6 database. A second virtual machine, a standard DS4 (8 vcpu, 28 Gb), has Apache Solr 7.x installed and is used to enable fast text based searches of the data in ERNIE. Automated processes are managed through the Jenkins Continuous Integration server and use custom ETL processes in Python to load and refresh data into the ERNIE schema. Secure access is managed through multiple factor authentication. Ssh tunnels are used to facilitate communications between virtual machines. Specifications, deployment scripts, and ETL scripts are available on a Github site~\cite{GithubERNIE2017}. \emph{[Need some text from DK and Avon!]}

\emph{Data sources} Data in ERNIE are derived from publicly available and commercial sources. Publicly available data are copied from the National Clinical Trials Database, the FDA Orange and Purple Books,  the United States Patent Office, and NIH ExPORTER. Leased data are acquired from Clarivate Analytics and consist of the Web of Science Core Collection and the Derwent Patent Citation Index. The data in ERNIE are stored in a relational schema that is also documented in Github. 

\emph{Data mining and digitization} A set of source documents relevant to the question being asked is assembled and mined for citations to scientific literature using manual or semi-manual methods. Manual extraction involved searching for text strings that represent references of interest and matching those to unique identifiers in PubMed and/or Web of Science. As an example, the medical review document relevant to approval of ipilimumab contained X references. Y of them were matched to PubMed identifiers using the PubMed either through its web interface or the E-utils API. This process was modified to take advantage of the Solr search engine. 

\emph{Amplification}

\emph{Network Calculations}

\emph{Visualizations}

\section*{Results and Discussion} 
\section*{Acknowledgments} Research and development reported in this publication was supported by Federal funds from the National Institute on Drug Abuse, National Institutes of Health, US Department of Health and Human Services, under Contract No. HHSN271201700053C. The content of this publication is solely the responsibility of the authors and does not necessarily represent the official views of the National Institutes of Health.


\section*{References}

\bibliography{chacko} 

\begin{figure}[!h]
%\begin{adjustwidth}{-0.5in}{0in} % Comment out/remove adjustwidth environment if table fits in text column.
\centering
\scalebox{0.99}
{
\begin{subfigure}{.5\textwidth}
  \centering
%  \includegraphics[width=.95\linewidth]{}
  \label{fig:sub1}
\end{subfigure}
\begin{subfigure}{.5\textwidth}
  \centering
 % \includegraphics[width=.95\linewidth]{}
  \label{fig:sub2}
\end{subfigure}
}
\caption{{\bf Intersecting Publications in Five Networks} Intersections were calculated across all five networks for the first generation of references (citing\_pmids) and 
as well as for the second generation of references (cited\_sids) and displayed as Venn diagrams. \emph{Left Panel.} No first generation publications are observed common to all five networks. A single publication is cited in four of five networks. \emph{Right Panel.} 14  publications are common to all five networks. Abbreviations: alem (Alemtuzumab), imat (Imatinib), nela (Nelarabine), ramu (Ramucirumab), suni(Sunitinib)}
\label{fig: test}
%\end{adjustwidth}
\end{figure}

\begin{figure}[!h]
%\begin{adjustwidth}{-0.5in}{0in} % Comment out/remove adjustwidth environment if table fits in text column.
\centering
%\scalebox{1.3}{
%\includegraphics[scale=0.1]{cy_core14_v2csv_5b.png}}
\caption{{\bf Core Publications in Networks}  The outer arcs of blue nodes identifies first generation publications (citing\_sid) for each therapeutic. Nodes in the inner ring are sized by a gradient proportion to total degree count with an upper limit of 30 and are colored by a gradient proportional to the number of drug connections (2 to 5). 14  publications are common to all five networks (Table 3) and are colored red. The remaining nodes in the inner ring connect to between 2 and 4 drugs each and are labeled accordingly. Abbreviations: alem (Alemtuzumab), imat (Imatinib), nela (Nelarabine), ramu (Ramucirumab), suni(Sunitinib).}
\label{fig2}
%\end{adjustwidth}
\end{figure}

\end{document}